\Title{Zahlw\"orter}

\Subst{n}{Zahlwort}{...w\"orter}{имя числительное}

\begin{longtable}{llll}
N & количественные & порядковые \\
1 & eins & erster \\
2 & zwei & zweiter \\
3 & drei & dritter \\
4 & vier & vierter \\
5 & f\"unf & f\"unfter \\
6 & sechs & sechster \\
7 & sieben & siebenter \\
8 & acht & achter \\
9 & neun & neunter \\
10 & zehn & zehnter \\
11 & elf & elfter \\
12 & zw\"olf & zw\"olfter \\
13 & dreizehn & dreizehnter \\
14 & vierzehn & vierzehnter \\
15 & f\"unfzehn & f\"unfzehnter \\
16 & sechzehn & sechzehnter \\
17 & siebzehn & siebzehnter \\
18 & achtzehn & achtzehnter \\
19 & neunzehn & neunzehnter \\
20 & zwanzig & zwanzigster \\
21 & einundzwanzig & einundzwanzigster \\
30 & drei\ss ig & drei\ss igster \\
40 & vierzig & vierzigster \\
100 & einhundert & einhundertster \\
125 & einhundertf\"unfundzwanzig & einhundertf\"unfundzwanzigster \\
655 & sechshundertf\"unfundf\"unfzig & sechshundertf\"unfundf\"unfzigster \\
1000 & eintausend & eintausendster \\
10000 & zehntausend & zehntausendster \\
100000 & einhunderttausend & einhunderttausendster \\
1000000 & ein Million & erster Million / einmillionster \\
\end{longtable}

Года исчисляются сотнями:

1981~---~neinzehnhunderteinundachtzig

Порядковые:

\begin{tabular}{ll}
1 & особый случай \\
2-19 & -ter \\
от 20 & -ster
\end{tabular}

Перечисление пунктов:

\begin{tabular}{ll}
во-первых & erstens \\
во-вторых & zweitens \\
в-третьих & drittens \\
в-четвертых & viertens
\end{tabular}

Наречия:

\begin{tabular}{ll}
однажды & einmal \\
дважды & zweimal \\
трижды & dreimal
\end{tabular}
