\Title{Lebensmittel}

\Subst{n}{Lebensmittel}{=}{продукт питания} \\
\Subst{n}{Essen}{---}{еда, трапеза} \\
\Subst{f}{Kost}{---}{еда, диета, рацион, пища (в т.ч. переносн.: духовная, для размышлений и т.д.)} \\
\Subst{f}{Speise}{-n}{еда, кушанье, блюдо} \\
\Subst{n}{Getr\"ank}{-e}{напиток}

\Subtitle{Eigenschaften}

\begin{longtable}{|l|l|}
\hline\endhead
\hline\endfoot
eiskalt & ледяной, очень холодный \\
kalt & холодный \\
warm & теплый \\
hei\ss & горячий, растопленный \\
\hline
d\"unn & жидкий, негустой, слабый (алкоголь) \\
w\"asserig & жидкий, водянистый \\
fl\"ussig & жидкий (физически) \\
z\"afl\"ussig & густой (о жидкости) \\
klebrig & вязкий, липкий \\
streubar & сыпучий \\
hart & твёрдый, черствый, крепкий (алкоголь) \\
m\"urbe & нежный (о мясе), рассыпчатый (о печенье), рыхлый (о тесте) \\
knusprig & хрустящий
\end{longtable}

\Subtitle{Geschmack}

\Subst{m}{Geschmack}{Geschm\"acke}{вкус}

\begin{longtable}{|l|l|}
\hline\endhead
\hline\endfoot
s\"u\ss & сладкий \\
sauer & кислый \\
salzig & соленый (содержащий соль) \\
gesalzen & соленый (посоленный) \\
bitter & горький \\
scharf & острый
\end{longtable}

\Subtitle{W\"urzen}

\Subst{f}{W\"urze}{-n}{специя, пряность}

\begin{longtable}{lll}
\TSubst{n}{Salz}{-e}{соль, поваренная соль} \\
\TASubst{jodiertes Salz}{йодированная соль} \\
\TSubst{m}{Zucker}{---}{сахар} \\
\TASubst{W\"urfelzucker}{пилёный, кусковой сахар, сахарная голова(?)} \\
\TSubst{m}{Pfeffer}{=}{перец} \\
\TSubst{m}{Paprika}{-s}{красный [стручковый] перец (Capsicum annuum L.)} \\
\TSubst{m}{Zimt}{-e}{корица} \\
\TSubst{f}{Muskatnu\ss}{...n\"usse}{мускатный орех} \\
\TSubst{m}{Koriander}{=}{кориандр} \\
\TSubst{n}{Lorbeerblatt}{...bl\"atter}{лавровый лист} \\
\TSubst{m}{Essig}{-e}{уксус} \\
\TSubst{m}{Knoblauch}{-e}{чеснок} \\
\TSubst{m}{Dill}{-e}{укроп} \\
\TSubst{f}{Petersilie}{...lien}{петрушка}
\end{longtable}

\Subtitle{Gem\"use}

\Subst{n}{Gem\"use}{=}{овощи, зелень}

\begin{longtable}{lll}
\TSubst{m}{Lauch}{-e}{лук (в т.ч. зеленый)} \\
\TSubst{f}{Zwiebel}{-n}{лук, луковица} \\
\TSubst{f}{Kartoffel}{-n}{картофель} \\
\TSubst{n}{Radieschen}{=}{редис} \\
\TSubst{f}{Gurke}{-n}{огурец} \\
\TSubst{m}{K\"urbis}{-se}{тыква} \\
\TSubst{m}{Walzenk\"urbis}{-se}{кабачок} \\
\TSubst{m}{Zucchino}{-ni}{кабачок, сорт: цукини} \\
\TSubst{f}{M\"ohre}{-n}{морковь} \\
\TSubst{f}{Karotte}{-n}{морковь, сорт: каротель}
\end{longtable}

\Subtitle{Obst und Beeren}

\Subst{n}{Obst}{---}{фрукты (собирательное определение)} \\
\Subst{f}{Beere}{-n}{ягода} \\
\Subst{f}{Frucht}{Fr\"uchte}{фрукт, плод}

\begin{longtable}{lll}
\TSubst{f}{Kirsche}{-n}{вишня} \\
\TSubst{f}{Erdbeere}{-n}{земляника} \\
\TSubst{f}{Gartenerdbeere}{-n}{клубника} \\
\TSubst{f}{Banane}{-n}{банан} \\
\TSubst{m}{Apfel}{\"Apfel}{яблоко} \\
\TSubst{f}{Ananas}{=}{ананас} \\
\TSubst{f}{Apfelsine}{-n}{апельсин} \\
\TSubst{f}{Orange}{-n}{апельсин} \\
\TSubst{f}{Mandarine}{-n}{мандарин} \\
\TSubst{m}{Pfirsich}{-e}{персик} \\
\TSubst{f}{Aprikose}{-n}{абрикос} \\
\TSubst{f}{---}{Weintrauben}{виноград} \\
\TSubst{f}{Birne}{---}{груша}
\end{longtable}
