\Title{Lebensmittel}

\Subst{n}{Lebensmittel}{=}{продукт питания} \\
\Subst{n}{Essen}{---}{еда, трапеза} \\
\Subst{f}{Kost}{---}{еда, диета, рацион, пища (в т.ч. переносн.: духовная, для размышлений и т.д.)} \\
\Subst{f}{Speise}{-n}{еда, кушанье, блюдо} \\
\Subst{n}{Getr\"ank}{-e}{напиток}

\Subtitle{Eigenschaften}

\begin{longtable}{|l|l|}
\hline\endhead
\hline\endfoot
eiskalt & ледяной, очень холодный \\
kalt & холодный \\
warm & теплый \\
hei\ss & горячий, растопленный \\
\hline
d\"unn & жидкий, негустой, слабый (алкоголь) \\
w\"asserig & жидкий, водянистый \\
fl\"ussig & жидкий (физически) \\
z\"afl\"ussig & густой (о жидкости) \\
klebrig & вязкий, липкий \\
streubar & сыпучий \\
hart & твёрдый, черствый, крепкий (алкоголь) \\
m\"urbe & нежный (о мясе), рассыпчатый (о печенье), рыхлый (о тесте) \\
knusprig & хрустящий
\end{longtable}

\Subtitle{Geschmack}

\Subst{m}{Geschmack}{Geschm\"acke}{вкус}

\begin{longtable}{|l|l|}
\hline\endhead
\hline\endfoot
s\"u\ss & сладкий \\
sauer & кислый \\
salzig & соленый (содержащий соль) \\
gesalzen & соленый (посоленный) \\
bitter & горький \\
scharf & острый
\end{longtable}
