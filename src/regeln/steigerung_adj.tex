\Title{Steigerung der Adjektive}

\Subst{f}{Steigerung}{-en}{образование степеней сравнения}

\begin{tabular}{ll}
Positiv & Adjektive \\
Komparativ & [Adjektivenstamm + er] \\
Superlativ & am [Adjektivenstamm + (e)sten] \\
oder & (bestimmten Artikel) [Adjektivenstamm + (e)ste]
\end{tabular}

zum Beispiel:

\begin{tabular}{|c|c|c|c|}
\hline
\multirow{2}{*}{Positiv} & \multirow{2}{*}{Komparativ} & \multicolumn{2}{c}{Superlativ} \\ \cline{3-4}
 &  & am ...+-(e)sten & der,die,das ...+-(e)ste \\
\hline
faul & fauler & am faulsten & der/die/das faulste \\
schnell & schneller & am schnellsten & der/die/das schnellste \\
\hline
\end{tabular}

Stamm mit -d, -\ss, -sch, -t, -tz, -x, -z $\rightarrow$ ...+(e)+...

zum Beispiel:

\begin{tabular}{|c|c|c|c|}
\hline
\multirow{2}{*}{Positiv} & \multirow{2}{*}{Komparativ} & \multicolumn{2}{c}{Superlativ} \\ \cline{3-4}
 &  & am ...+-(e)sten & der,die,das ...+-(e)ste \\
\hline
breit & breiter & am breitesten & der/die/das breiteste \\
h\"ubsch & h\"ubscher & am h\"ubschesten & der/die/das h\"ubscheste \\
s\"u\ss & s\"u\ss er & am s\"u\ss esten & der/die/das s\"u\ss este \\
weit & weiter & am weitesten & der/die/das weiteste \\
\hline
\end{tabular}

also: blind, bl\"od, dicht, echt, fest, fett, feucht, fies, fix, glatt, hei\ss, laut, leicht, leise, mies, mild, m\"ude, nett, rasch, sanft, satt, schlecht, sp\"at, spitz, stolz, weise, wild, zart

Einige Adjektive bilden ihre Steigerungsformen mit einem Umlaut:

\begin{tabular}{|c|c|c|}
\hline
Positiv & Komparativ & Superlativ \\
\hline
alt & \"alter & am \"altesten \\
arm & \"armer & am \"armsten \\
gro\ss & gr\"o\ss er & am gr\"o\ss ten \\
jung & j\"unger & am j\"ungsten \\
\hline
\end {tabular}

Weitere Adjektive sind:

dumm, gesund, grob, hart, kalt, klug, krank, kurz, lang, rot, scharf, stark, schwach, warm. 

Unregelm\"a\ss ige Ajektive:
\begin{longtable}{|c|c|c|}
\hline
Positiv & Komparativ & Superlativ \\
\hline\endhead
\hline\endfoot
gern & lieber & am liebsten \\
gut & besser & am besten \\
hoch & h\"oher & am h\"ochsten \\
nah & n\"aher & am n\"achsten \\
viel & mehr & am meisten \\
\end{longtable}
