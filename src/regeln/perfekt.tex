\Title{Perfekt}

Perfekt = \begin{tabular}{l}
haben \\
sein
\end{tabular} + Partizip II

Das Perfekt mit dem Hilfsverb "haben" bilden:
\begin{itemize}
\item alle Verben mit Akkusativ-Erg\"anzung:
\begin{itemize}
	\item Er liebt mich noch heute. - Damals habe ich ihn auch geliebt.
	\item Thomas liest ein Buch. - Thomas hat ein Buch gelesen.
	\item Hein gibt t\"aglich sehr viel Geld aus. - Hein hat t\"aglich sehr viel Geld ausgegeben.
\end{itemize}
\item alle reflexive Verben:
\begin{itemize}
	\item Er w\"ascht sich selten. - Heute hat er sich auch noch nicht gewaschen.
	\item Du erk\"altest dich noch. - Siehst du, du hast dich schon erk\"altet.
	\item Beeil dich! - Warum, du hast dich doch auch nicht beeilt.
\end{itemize}
\item alle Modalverben als Vollverb (ihr Gebrauch ist aber selten!!):
\begin{itemize}
	\item Das habe ich nicht gewollt. - Die Arbeit hat er nicht machen wollen.
	\item Der Sch\"uler hat die Aufgabe nicht gekonnt. - Der Sch\"uler hat es nicht machen können.
	\item So viele Hausaufgaben hast du nicht machen m\"ussen.
\end{itemize}
\item die meisten anderen Verben:
\begin{itemize}
	\item Mein Nachbar hilft mir nicht. - Aber ich habe ihm immer geholfen.
	\item Gibst du mir ein Bonbon ab? - Ich habe dir gestern auch eins abgegeben.
	\item Heute regnet es zum Gl\"uck nicht. - Gestern hat es den ganzen Tag geregnet.
\end{itemize}
\end{itemize}

Seitenanfang

Das Perfekt mit dem Hilfsverb "sein" bilden:
\begin{itemize}
\item alle Verben der Ortsver\"anderung: \\
\begin{tabular}{|l|l|l|l|l|}
\hline
Infinitiv & Position 1 & Verb 1 & Mittelfeld & Verb 2 \\
\hline
gehen & Mein Kollege & ist & heute fr\"uher nach Hause & gegangen. \\
an|kommen & Unser Zug & ist & heute mal wieder zu sp\"at & angekommen. \\
fahren & Gestern & sind & wir mit dem Fahrrad nach Ulm & gefahren. \\
\hline
\end{tabular}
\item alle Verben der Zustands\"anderung: \\
\begin{tabular}{|l|l|l|l|l|}
\hline
Infinitiv & Position 1 & Verb 1 & Mittelfeld & Verb 2 \\
\hline
auf|stehen & Ich & bin & heute Morgen sehr fr\"uh & aufgestanden. \\
auf|wachsen & Meine Frau & ist & in einem kleinen Dorf bei Ulm & aufgewachsen. \\
einschlafen & Endlich & ist & das kranke Kind wieder & eingeschlafen. \\
sterben & Ihr Mann & ist & schon mit 43 Jahren & gestorben. \\
wachsen & Was & sind & deine Kinder schon & gewachsen. \\
\hline
\end{tabular}
\item folgende Verben: \\
\begin{tabular} {|l|l|l|l|l|}
\hline
Infinitiv & Position 1 & Verb 1 & Mittelfeld & Verb 2 \\
\hline
bleiben & Mein Freund & ist & gestern sehr lang bei uns & geblieben. \\
gelingen & Mir & ist & endlich mein Experiment & gelungen. \\
geschehen & Was & ist & gestern eigentlich auf der Party & geschehen? \\
passieren & Gestern & ist & etwas Schreckliches & passiert. \\
sein & ~ & Seid & ihr auch schon mal in der Schweiz & gewesen? \\
werden & Das Kind & ist & heute 8 Jahre alt & geworden. \\
\hline
\end{tabular}
\end{itemize}
