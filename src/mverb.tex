\Title{Modal Verben}

\begin{tabular}{ll}
Возможность: & k\"onnen / d\"urfen \\
Долженствование: & m\"ussen / sollen \\
Желание: & wollen / m\"ogen \\
Предположение: & m\"ogen / k\"onnen / d\"urfen / m\"ussen
\end{tabular}

\begin{longtable}{|l|p{12cm}|}
\hline
\multicolumn{2}{|l|}{Возможность: k\"onnen / d\"urfen} \\
\hline
k\"onnen & \begin{tabular}{lp{5cm}p{5cm}}
1) & \multicolumn{2}{p{11cm}}{мочь, уметь, быть в состоянии (возможность по объективным обстоятельствам)} \\
 & Sie kann deutsch sprechen. & Она может (умеет) говорить по-немецки. \\
 & Er kann nicht kommen. & Он не может прийти.
\end{tabular} \\
\hline
d\"urfen & \begin{tabular}{lp{5cm}p{5cm}}
1) & \multicolumn{2}{p{11cm}}{мочь~---~сметь, иметь разрешение (возможность, основанная на "чужой воле")} \\
 & Darf ich fragen? & Можно (разрешите) спросить? \\
2) & \multicolumn{2}{p{11cm}}{при отрицании выражает запрещение~---~"нельзя", "не разрешается"} \\
 & Man darf hier nicht laut sprechen. &	Здесь нельзя громко разговаривать.
\end{tabular} \\
\hline
\multicolumn{2}{|l|}{Долженствование: m\"ussen / sollen} \\
\hline
m\"ussen & \begin{tabular}{lp{5cm}p{5cm}}
1) & \multicolumn{2}{p{11cm}}{долженствование, необходимость, потребность, осознанный долг} \\
 & Ich muss nach Hause gehen. & Я должен (мне нужно) идти домой. \\
2) & \multicolumn{2}{p{11cm}}{при отрицании "m\"ussen" часто заменяется глаголом "brauchen + zu Infinitiv)} \\
 & Sie brauchen nicht zu kommen. & Вам не нужно (нет нужды, можно не) приходить.
\end{tabular} \\
\hline
sollen & \begin{tabular}{lp{5cm}p{5cm}}
1) & \multicolumn{2}{p{11cm}}{долженствование, основанное на "чужой воле"~---~приказ, поручение, распоряжение} \\
 & Sie sollen diese Arbeit heute machen. & Вы должны сделать эту работу сегодня. \\
2) & \multicolumn{2}{p{11cm}}{в вопросе (прямом или косвенном) не переводится (выражает "запрос инструкции, распоряжения")} \\
 & Soll ich den Satz \"ubersetzen? & Мне (следует) перевести это предложение? \\
 & Er fragt, ob er den Satz übersetzen soll. & Он спрашивает, переводить ли ему это предложение.
\end{tabular} \\
\hline
\multicolumn{2}{|l|}{Желание: wollen / m\"ogen} \\
\hline
wollen & \begin{tabular}{lp{5cm}p{5cm}}
1) & \multicolumn{2}{p{11cm}}{хотеть, намереваться, собираться} \\
 & Er will Arzt werden. & Он хочет стать врачом. \\
2) & \multicolumn{2}{p{11cm}}{приглашение к совместному действию} \\
 & Wollen wir ins Kino gehen! (= Gehen wir ins Kino!) & Давайте пойдем в кино!
\end{tabular} \\
\hline
m\"ogen & \begin{tabular}{lp{5cm}p{5cm}}
1) & \multicolumn{2}{p{11cm}}{"хотел бы"~---~в форме m\"ochte (вежливо выраженное желание в настоящем времени)} \\
 & Ich m\"ochte meinen Bruder mitnehmen. & Я хотел бы взять с собой своего брата. \\
2) & \multicolumn{2}{p{11cm}}{любить, нравиться~---~в самостоятельном значении (при употреблении без сопровождающего инфинитива)} \\
 & Ich mag Eis. Er mag Fisch nicht. & Я люблю мороженое. Он не любит рыбу.
\end{tabular} \\
\hline
\multicolumn{2}{|l|}{Предположение: m\"ogen / k\"onnen / d\"urfen / m\"ussen} \\
\hline
m\"ogen & \begin{tabular}{p{5.4cm}p{5.4cm}}
\multicolumn{2}{p{11cm}}{может быть, возможно (допущение возможности, неуверенное предположение)} \\
Das mag richtig sein. & Может быть, это правильно.
\end{tabular} \\
\hline
k\"onnen & \begin{tabular}{p{5.4cm}p{5.4cm}}
\multicolumn{2}{p{11cm}}{возможно (предположение, основанное на объективной возможности)} \\
Das kann richtig sein. & Возможно, это правильно.
\end{tabular} \\
\hline
d\"urfen & \begin{tabular}{p{5.4cm}p{5.4cm}}
\multicolumn{2}{p{11cm}}{в форме сослагательного наклонения~---~d\"urfte~---~вероятно, должно быть (предположение более высокой степени уверенности, чем предположение, выраженное глаголами k\"onnen и m\"ogen)} \\
Das d\"urfte richtig sein. & Вероятно, это правильно.
\end{tabular} \\
\hline
m\"ussen & \begin{tabular}{p{5.4cm}p{5.4cm}} 
\multicolumn{2}{p{11cm}}{должно быть, наверное, определенно (предположение, граничащее с уверенностью)} \\
Das muss richtig sein. & Должно быть, это правильно.
\end{tabular} \\
\hline
\end{longtable}

\begin{tabular}{cccccccc}
 & k\"onnen & d\"urfen & m\"ussen & sollen & wollen & m\"ogen \\
ich & kann & darf & muss & soll & will & mag/m\"ochte \\
du & kannst & darfst & musst & sollst & willst & magst/m\"ochtest \\
er/sie/es & kann & darf & muss & soll & will & mag/m\"ochte \\
wir & k\"onnen & d\"urfen & m\"ussen & sollen & wollen & m\"ogen/m\"ochten \\
ihr & k\"onnt & d\"urft & m\"usst & sollt & wollt & m\"ogt/m\"ochtet \\
sie/Sie & k\"onnen & d\"urfen & m\"ussen & sollen & wollen & m\"ogen/m\"ochten
\end{tabular}

Местоимение man в сочетании с модальными глаголами переводится безличными конструкциями:

\begin{tabular}{rlrl}
man kann & ~---~можно &	man kann nicht & ~---~нельзя, невозможно \\
man darf & ~---~можно, разрешается & man darf nicht & ~---~нельзя, не разрешается \\
man muss & ~---~нужно, необходимо & man muss nicht & ~---~не нужно, не обязательно \\
man soll & ~---~следует, надо & man soll nicht & ~---~не следует
\end{tabular}
